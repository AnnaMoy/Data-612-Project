% Options for packages loaded elsewhere
\PassOptionsToPackage{unicode}{hyperref}
\PassOptionsToPackage{hyphens}{url}
%
\documentclass[
]{article}
\usepackage{amsmath,amssymb}
\usepackage{iftex}
\ifPDFTeX
  \usepackage[T1]{fontenc}
  \usepackage[utf8]{inputenc}
  \usepackage{textcomp} % provide euro and other symbols
\else % if luatex or xetex
  \usepackage{unicode-math} % this also loads fontspec
  \defaultfontfeatures{Scale=MatchLowercase}
  \defaultfontfeatures[\rmfamily]{Ligatures=TeX,Scale=1}
\fi
\usepackage{lmodern}
\ifPDFTeX\else
  % xetex/luatex font selection
\fi
% Use upquote if available, for straight quotes in verbatim environments
\IfFileExists{upquote.sty}{\usepackage{upquote}}{}
\IfFileExists{microtype.sty}{% use microtype if available
  \usepackage[]{microtype}
  \UseMicrotypeSet[protrusion]{basicmath} % disable protrusion for tt fonts
}{}
\makeatletter
\@ifundefined{KOMAClassName}{% if non-KOMA class
  \IfFileExists{parskip.sty}{%
    \usepackage{parskip}
  }{% else
    \setlength{\parindent}{0pt}
    \setlength{\parskip}{6pt plus 2pt minus 1pt}}
}{% if KOMA class
  \KOMAoptions{parskip=half}}
\makeatother
\usepackage{xcolor}
\usepackage[margin=1in]{geometry}
\usepackage{color}
\usepackage{fancyvrb}
\newcommand{\VerbBar}{|}
\newcommand{\VERB}{\Verb[commandchars=\\\{\}]}
\DefineVerbatimEnvironment{Highlighting}{Verbatim}{commandchars=\\\{\}}
% Add ',fontsize=\small' for more characters per line
\usepackage{framed}
\definecolor{shadecolor}{RGB}{248,248,248}
\newenvironment{Shaded}{\begin{snugshade}}{\end{snugshade}}
\newcommand{\AlertTok}[1]{\textcolor[rgb]{0.94,0.16,0.16}{#1}}
\newcommand{\AnnotationTok}[1]{\textcolor[rgb]{0.56,0.35,0.01}{\textbf{\textit{#1}}}}
\newcommand{\AttributeTok}[1]{\textcolor[rgb]{0.13,0.29,0.53}{#1}}
\newcommand{\BaseNTok}[1]{\textcolor[rgb]{0.00,0.00,0.81}{#1}}
\newcommand{\BuiltInTok}[1]{#1}
\newcommand{\CharTok}[1]{\textcolor[rgb]{0.31,0.60,0.02}{#1}}
\newcommand{\CommentTok}[1]{\textcolor[rgb]{0.56,0.35,0.01}{\textit{#1}}}
\newcommand{\CommentVarTok}[1]{\textcolor[rgb]{0.56,0.35,0.01}{\textbf{\textit{#1}}}}
\newcommand{\ConstantTok}[1]{\textcolor[rgb]{0.56,0.35,0.01}{#1}}
\newcommand{\ControlFlowTok}[1]{\textcolor[rgb]{0.13,0.29,0.53}{\textbf{#1}}}
\newcommand{\DataTypeTok}[1]{\textcolor[rgb]{0.13,0.29,0.53}{#1}}
\newcommand{\DecValTok}[1]{\textcolor[rgb]{0.00,0.00,0.81}{#1}}
\newcommand{\DocumentationTok}[1]{\textcolor[rgb]{0.56,0.35,0.01}{\textbf{\textit{#1}}}}
\newcommand{\ErrorTok}[1]{\textcolor[rgb]{0.64,0.00,0.00}{\textbf{#1}}}
\newcommand{\ExtensionTok}[1]{#1}
\newcommand{\FloatTok}[1]{\textcolor[rgb]{0.00,0.00,0.81}{#1}}
\newcommand{\FunctionTok}[1]{\textcolor[rgb]{0.13,0.29,0.53}{\textbf{#1}}}
\newcommand{\ImportTok}[1]{#1}
\newcommand{\InformationTok}[1]{\textcolor[rgb]{0.56,0.35,0.01}{\textbf{\textit{#1}}}}
\newcommand{\KeywordTok}[1]{\textcolor[rgb]{0.13,0.29,0.53}{\textbf{#1}}}
\newcommand{\NormalTok}[1]{#1}
\newcommand{\OperatorTok}[1]{\textcolor[rgb]{0.81,0.36,0.00}{\textbf{#1}}}
\newcommand{\OtherTok}[1]{\textcolor[rgb]{0.56,0.35,0.01}{#1}}
\newcommand{\PreprocessorTok}[1]{\textcolor[rgb]{0.56,0.35,0.01}{\textit{#1}}}
\newcommand{\RegionMarkerTok}[1]{#1}
\newcommand{\SpecialCharTok}[1]{\textcolor[rgb]{0.81,0.36,0.00}{\textbf{#1}}}
\newcommand{\SpecialStringTok}[1]{\textcolor[rgb]{0.31,0.60,0.02}{#1}}
\newcommand{\StringTok}[1]{\textcolor[rgb]{0.31,0.60,0.02}{#1}}
\newcommand{\VariableTok}[1]{\textcolor[rgb]{0.00,0.00,0.00}{#1}}
\newcommand{\VerbatimStringTok}[1]{\textcolor[rgb]{0.31,0.60,0.02}{#1}}
\newcommand{\WarningTok}[1]{\textcolor[rgb]{0.56,0.35,0.01}{\textbf{\textit{#1}}}}
\usepackage{graphicx}
\makeatletter
\newsavebox\pandoc@box
\newcommand*\pandocbounded[1]{% scales image to fit in text height/width
  \sbox\pandoc@box{#1}%
  \Gscale@div\@tempa{\textheight}{\dimexpr\ht\pandoc@box+\dp\pandoc@box\relax}%
  \Gscale@div\@tempb{\linewidth}{\wd\pandoc@box}%
  \ifdim\@tempb\p@<\@tempa\p@\let\@tempa\@tempb\fi% select the smaller of both
  \ifdim\@tempa\p@<\p@\scalebox{\@tempa}{\usebox\pandoc@box}%
  \else\usebox{\pandoc@box}%
  \fi%
}
% Set default figure placement to htbp
\def\fps@figure{htbp}
\makeatother
\setlength{\emergencystretch}{3em} % prevent overfull lines
\providecommand{\tightlist}{%
  \setlength{\itemsep}{0pt}\setlength{\parskip}{0pt}}
\setcounter{secnumdepth}{-\maxdimen} % remove section numbering
\usepackage{bookmark}
\IfFileExists{xurl.sty}{\usepackage{xurl}}{} % add URL line breaks if available
\urlstyle{same}
\hypersetup{
  pdftitle={Data 613 - Project 1},
  pdfauthor={Anna Moy \& Natalie Kalukeerthie},
  hidelinks,
  pdfcreator={LaTeX via pandoc}}

\title{Data 613 - Project 1}
\author{Anna Moy \& Natalie Kalukeerthie}
\date{2025-06-08}

\begin{document}
\maketitle

\begin{Shaded}
\begin{Highlighting}[]
\CommentTok{\# Load library}
\FunctionTok{library}\NormalTok{(reshape2)}
\end{Highlighting}
\end{Shaded}

\section{Overview}\label{overview}

\textbf{Briefly describe the recommender system that you're going to
build out from a business perspective, e.g.~``This system recommends
data science books to readers.''}

For this project, we are building a recommender system for Yelp-style
restaurant reviews. Users rate restaurants from 1 to 5, but not every
user rates every restaurant (in this project our item will be
`restaurants'). Our goal is to predict missing ratings, using two simple
models:

\begin{itemize}
\item
  A \textbf{Raw Average Model} that predicts all ratings using the
  global mean.
\item
  A \textbf{Baseline Model} that adjusts predictions based on both user
  and restaurant (item) biases.
\end{itemize}

Our core idea is that evaluating how well each model performs (using
RMSE), we aim to learn how bias-aware predictions improve recommendation
accuracy. We simulate reviews from 6 Yelp users (Alice, Ben, Cindy,
David, Ella, Frank) for 5 restaurants (Pasta Place, Sushi Spot, Burger
Barn, Curry Corner, Taco Town). Keep in mind that some users have not
rated every restaurant.

\begin{Shaded}
\begin{Highlighting}[]
\CommentTok{\# Create Dataset with 6 users and 5 restaurants}
\NormalTok{ratings\_df }\OtherTok{\textless{}{-}} \FunctionTok{data.frame}\NormalTok{(}
  \AttributeTok{user =} \FunctionTok{c}\NormalTok{(}
    \FunctionTok{rep}\NormalTok{(}\StringTok{"Alice"}\NormalTok{, }\DecValTok{5}\NormalTok{),}
    \FunctionTok{rep}\NormalTok{(}\StringTok{"Ben"}\NormalTok{, }\DecValTok{5}\NormalTok{),}
    \FunctionTok{rep}\NormalTok{(}\StringTok{"Cindy"}\NormalTok{, }\DecValTok{5}\NormalTok{),}
    \FunctionTok{rep}\NormalTok{(}\StringTok{"David"}\NormalTok{, }\DecValTok{5}\NormalTok{),}
    \FunctionTok{rep}\NormalTok{(}\StringTok{"Ella"}\NormalTok{, }\DecValTok{5}\NormalTok{),}
    \FunctionTok{rep}\NormalTok{(}\StringTok{"Frank"}\NormalTok{, }\DecValTok{5}\NormalTok{)}
\NormalTok{  ),}
  \AttributeTok{restaurant =} \FunctionTok{rep}\NormalTok{(}\FunctionTok{c}\NormalTok{(}\StringTok{"Pasta\_Place"}\NormalTok{, }\StringTok{"Sushi\_Spot"}\NormalTok{, }\StringTok{"Burger\_Barn"}\NormalTok{, }\StringTok{"Curry\_Corner"}\NormalTok{, }\StringTok{"Taco\_Town"}\NormalTok{), }\DecValTok{6}\NormalTok{),}
  \AttributeTok{rating =} \FunctionTok{c}\NormalTok{(}
    \DecValTok{5}\NormalTok{, }\ConstantTok{NA}\NormalTok{, }\DecValTok{4}\NormalTok{, }\ConstantTok{NA}\NormalTok{, }\DecValTok{4}\NormalTok{,      }\CommentTok{\# Alice}
    \DecValTok{4}\NormalTok{, }\DecValTok{3}\NormalTok{, }\DecValTok{5}\NormalTok{, }\DecValTok{3}\NormalTok{, }\DecValTok{4}\NormalTok{,        }\CommentTok{\# Ben}
    \DecValTok{4}\NormalTok{, }\DecValTok{2}\NormalTok{, }\ConstantTok{NA}\NormalTok{, }\ConstantTok{NA}\NormalTok{, }\DecValTok{3}\NormalTok{,      }\CommentTok{\# Cindy}
    \DecValTok{2}\NormalTok{, }\DecValTok{2}\NormalTok{, }\DecValTok{3}\NormalTok{, }\DecValTok{1}\NormalTok{, }\DecValTok{2}\NormalTok{,        }\CommentTok{\# David}
    \DecValTok{4}\NormalTok{, }\ConstantTok{NA}\NormalTok{, }\DecValTok{5}\NormalTok{, }\DecValTok{4}\NormalTok{, }\DecValTok{5}\NormalTok{,       }\CommentTok{\# Ella}
    \DecValTok{4}\NormalTok{, }\DecValTok{2}\NormalTok{, }\DecValTok{5}\NormalTok{, }\DecValTok{4}\NormalTok{, }\DecValTok{4}         \CommentTok{\# Frank}
\NormalTok{  )}
\NormalTok{)}

\NormalTok{ratings\_df}
\end{Highlighting}
\end{Shaded}

\begin{verbatim}
##     user   restaurant rating
## 1  Alice  Pasta_Place      5
## 2  Alice   Sushi_Spot     NA
## 3  Alice  Burger_Barn      4
## 4  Alice Curry_Corner     NA
## 5  Alice    Taco_Town      4
## 6    Ben  Pasta_Place      4
## 7    Ben   Sushi_Spot      3
## 8    Ben  Burger_Barn      5
## 9    Ben Curry_Corner      3
## 10   Ben    Taco_Town      4
## 11 Cindy  Pasta_Place      4
## 12 Cindy   Sushi_Spot      2
## 13 Cindy  Burger_Barn     NA
## 14 Cindy Curry_Corner     NA
## 15 Cindy    Taco_Town      3
## 16 David  Pasta_Place      2
## 17 David   Sushi_Spot      2
## 18 David  Burger_Barn      3
## 19 David Curry_Corner      1
## 20 David    Taco_Town      2
## 21  Ella  Pasta_Place      4
## 22  Ella   Sushi_Spot     NA
## 23  Ella  Burger_Barn      5
## 24  Ella Curry_Corner      4
## 25  Ella    Taco_Town      5
## 26 Frank  Pasta_Place      4
## 27 Frank   Sushi_Spot      2
## 28 Frank  Burger_Barn      5
## 29 Frank Curry_Corner      4
## 30 Frank    Taco_Town      4
\end{verbatim}

We pivot the data into a matrix format (User-Item Matrix), which helps
visualize missing ratings.

\begin{Shaded}
\begin{Highlighting}[]
\CommentTok{\#Transform data frame into a user{-}item matrix}

\CommentTok{\#pivot the long format into a matrix for user, item and ratings}
\NormalTok{matrix }\OtherTok{\textless{}{-}} \FunctionTok{dcast}\NormalTok{(ratings\_df, user }\SpecialCharTok{\textasciitilde{}}\NormalTok{ restaurant, }\AttributeTok{value.var =} \StringTok{"rating"}\NormalTok{) }

\CommentTok{\# row labels will be the based on user}
\FunctionTok{rownames}\NormalTok{(matrix) }\OtherTok{\textless{}{-}}\NormalTok{ matrix}\SpecialCharTok{$}\NormalTok{user}

\CommentTok{\# remove the first column since it is repeating duplicate value}
\NormalTok{matrix }\OtherTok{\textless{}{-}}\NormalTok{ matrix[, }\SpecialCharTok{{-}}\DecValTok{1}\NormalTok{]}

\CommentTok{\#print the matrix}
\FunctionTok{print}\NormalTok{(matrix)}
\end{Highlighting}
\end{Shaded}

\begin{verbatim}
##       Burger_Barn Curry_Corner Pasta_Place Sushi_Spot Taco_Town
## Alice           4           NA           5         NA         4
## Ben             5            3           4          3         4
## Cindy          NA           NA           4          2         3
## David           3            1           2          2         2
## Ella            5            4           4         NA         5
## Frank           5            4           4          2         4
\end{verbatim}

Next, we will split the data into a test (20\% of data) and training
(80\% of data) dataset

\begin{Shaded}
\begin{Highlighting}[]
\FunctionTok{set.seed}\NormalTok{(}\DecValTok{42}\NormalTok{)}
\CommentTok{\# Take the data and split into training 80\% and testing 20\% dataframe}
\NormalTok{train\_indices }\OtherTok{\textless{}{-}} \FunctionTok{sample}\NormalTok{(}\DecValTok{1}\SpecialCharTok{:}\FunctionTok{nrow}\NormalTok{(ratings\_df), }\AttributeTok{size =} \FloatTok{0.8} \SpecialCharTok{*} \FunctionTok{nrow}\NormalTok{(ratings\_df))}

\NormalTok{train\_df }\OtherTok{\textless{}{-}}\NormalTok{ ratings\_df[train\_indices, ]}
\NormalTok{test\_df }\OtherTok{\textless{}{-}}\NormalTok{ ratings\_df[}\SpecialCharTok{{-}}\NormalTok{train\_indices, ]}

\NormalTok{train\_df}
\end{Highlighting}
\end{Shaded}

\begin{verbatim}
##     user   restaurant rating
## 17 David   Sushi_Spot      2
## 5  Alice    Taco_Town      4
## 1  Alice  Pasta_Place      5
## 25  Ella    Taco_Town      5
## 10   Ben    Taco_Town      4
## 4  Alice Curry_Corner     NA
## 18 David  Burger_Barn      3
## 30 Frank    Taco_Town      4
## 15 Cindy    Taco_Town      3
## 7    Ben   Sushi_Spot      3
## 27 Frank   Sushi_Spot      2
## 29 Frank Curry_Corner      4
## 14 Cindy Curry_Corner     NA
## 22  Ella   Sushi_Spot     NA
## 3  Alice  Burger_Barn      4
## 9    Ben Curry_Corner      3
## 23  Ella  Burger_Barn      5
## 11 Cindy  Pasta_Place      4
## 20 David    Taco_Town      2
## 19 David Curry_Corner      1
## 13 Cindy  Burger_Barn     NA
## 12 Cindy   Sushi_Spot      2
## 2  Alice   Sushi_Spot     NA
## 24  Ella Curry_Corner      4
\end{verbatim}

\begin{Shaded}
\begin{Highlighting}[]
\NormalTok{test\_df}
\end{Highlighting}
\end{Shaded}

\begin{verbatim}
##     user  restaurant rating
## 6    Ben Pasta_Place      4
## 8    Ben Burger_Barn      5
## 16 David Pasta_Place      2
## 21  Ella Pasta_Place      4
## 26 Frank Pasta_Place      4
## 28 Frank Burger_Barn      5
\end{verbatim}

Our first model will be the raw average predictor, this model predicts
the same score (the global mean) for all missing values.

\begin{Shaded}
\begin{Highlighting}[]
\CommentTok{\# Find the Raw average and calculate RMSE}

\CommentTok{\# Raw Average on Training set and ignore NA}
\NormalTok{global\_mean }\OtherTok{\textless{}{-}} \FunctionTok{mean}\NormalTok{(train\_df}\SpecialCharTok{$}\NormalTok{rating, }\AttributeTok{na.rm =} \ConstantTok{TRUE}\NormalTok{)}


\CommentTok{\# Add the raw avg onto the training and test dataset}
\NormalTok{train\_df}\SpecialCharTok{$}\NormalTok{raw\_avg }\OtherTok{\textless{}{-}}\NormalTok{ global\_mean}
\NormalTok{test\_df}\SpecialCharTok{$}\NormalTok{raw\_avg }\OtherTok{\textless{}{-}}\NormalTok{ global\_mean}

\CommentTok{\# RMSE function}
\NormalTok{rmse }\OtherTok{\textless{}{-}} \ControlFlowTok{function}\NormalTok{(actual, predicted) \{}
  \FunctionTok{sqrt}\NormalTok{(}\FunctionTok{mean}\NormalTok{((actual }\SpecialCharTok{{-}}\NormalTok{ predicted)}\SpecialCharTok{\^{}}\DecValTok{2}\NormalTok{))}
\NormalTok{\}}

\NormalTok{train\_fil }\OtherTok{\textless{}{-}}\NormalTok{ train\_df[}\SpecialCharTok{!}\FunctionTok{is.na}\NormalTok{(train\_df}\SpecialCharTok{$}\NormalTok{rating), ]}
\NormalTok{test\_fil }\OtherTok{\textless{}{-}}\NormalTok{ test\_df[}\SpecialCharTok{!}\FunctionTok{is.na}\NormalTok{(test\_df}\SpecialCharTok{$}\NormalTok{rating), ]}


\CommentTok{\# Calculate RSME which takes the difference of ratings {-} avg mean for training and test set}
\NormalTok{rmse\_train\_raw }\OtherTok{\textless{}{-}} \FunctionTok{rmse}\NormalTok{(train\_fil}\SpecialCharTok{$}\NormalTok{rating, train\_fil}\SpecialCharTok{$}\NormalTok{raw\_avg)}
\NormalTok{rmse\_test\_raw }\OtherTok{\textless{}{-}} \FunctionTok{rmse}\NormalTok{(test\_fil}\SpecialCharTok{$}\NormalTok{rating, test\_fil}\SpecialCharTok{$}\NormalTok{raw\_avg)}

\NormalTok{rmse\_train\_raw}
\end{Highlighting}
\end{Shaded}

\begin{verbatim}
## [1] 1.133719
\end{verbatim}

\begin{Shaded}
\begin{Highlighting}[]
\NormalTok{rmse\_test\_raw}
\end{Highlighting}
\end{Shaded}

\begin{verbatim}
## [1] 1.182748
\end{verbatim}

Our second model 2 is the \textbf{Baseline Model} assumes that each user
and item (restaurant) has a tendency to rate higher or lower than
average. This model adjusts predictions using:

\begin{itemize}
\item
  \textbf{User bias}: How much a user rates higher/lower than average.
\item
  \textbf{Item bias}: How popular (or unpopular) a restaurant is
  compared to others.
\end{itemize}

We predict the rating as is: Predicted Rating = Global Mean + User Bias
+ Item Bias

This is a considered a collaborative filtering approach, where bias
correction leads to more personalized predictions.

To prevent overfitting, we applied regularization to user and item bias
estimates. This penalizes users or restaurants with very few ratings,
reducing the risk of exaggerated bias values. For example, if a user has
rated only one item extremely high, regularization ensures that this
does not overly skew their predicted ratings across the system.

\begin{Shaded}
\begin{Highlighting}[]
\CommentTok{\#Finds out the bias values for all columns and rows in grid}

\FunctionTok{library}\NormalTok{(dplyr)}
\end{Highlighting}
\end{Shaded}

\begin{verbatim}
## 
## Attaching package: 'dplyr'
\end{verbatim}

\begin{verbatim}
## The following objects are masked from 'package:stats':
## 
##     filter, lag
\end{verbatim}

\begin{verbatim}
## The following objects are masked from 'package:base':
## 
##     intersect, setdiff, setequal, union
\end{verbatim}

\begin{Shaded}
\begin{Highlighting}[]
\CommentTok{\# Set regularization strength}
\NormalTok{lambda }\OtherTok{\textless{}{-}} \DecValTok{10}

\CommentTok{\# Compute regularized user bias}
\NormalTok{user\_bias\_df }\OtherTok{\textless{}{-}}\NormalTok{ train\_df[}\SpecialCharTok{!}\FunctionTok{is.na}\NormalTok{(train\_df}\SpecialCharTok{$}\NormalTok{rating), ] }\SpecialCharTok{\%\textgreater{}\%}
  \FunctionTok{group\_by}\NormalTok{(user) }\SpecialCharTok{\%\textgreater{}\%}
  \FunctionTok{summarise}\NormalTok{(}
    \AttributeTok{num\_ratings =} \FunctionTok{n}\NormalTok{(),}
    \AttributeTok{sum\_diff =} \FunctionTok{sum}\NormalTok{(rating }\SpecialCharTok{{-}}\NormalTok{ global\_mean),}
    \AttributeTok{user\_bias =}\NormalTok{ sum\_diff }\SpecialCharTok{/}\NormalTok{ (num\_ratings }\SpecialCharTok{+}\NormalTok{ lambda)}
\NormalTok{  )}

\CommentTok{\# Compute regularized restaurant bias}
\NormalTok{restaurant\_bias\_df }\OtherTok{\textless{}{-}}\NormalTok{ train\_df[}\SpecialCharTok{!}\FunctionTok{is.na}\NormalTok{(train\_df}\SpecialCharTok{$}\NormalTok{rating), ] }\SpecialCharTok{\%\textgreater{}\%}
  \FunctionTok{group\_by}\NormalTok{(restaurant) }\SpecialCharTok{\%\textgreater{}\%}
  \FunctionTok{summarise}\NormalTok{(}
    \AttributeTok{num\_ratings =} \FunctionTok{n}\NormalTok{(),}
    \AttributeTok{sum\_diff =} \FunctionTok{sum}\NormalTok{(rating }\SpecialCharTok{{-}}\NormalTok{ global\_mean),}
    \AttributeTok{restaurant\_bias =}\NormalTok{ sum\_diff }\SpecialCharTok{/}\NormalTok{ (num\_ratings }\SpecialCharTok{+}\NormalTok{ lambda)}
\NormalTok{  )}

\CommentTok{\# Find the mean for all bia users in training data}
\CommentTok{\#user\_avg \textless{}{-} aggregate(rating \textasciitilde{} user, data = train\_df, mean)}
\CommentTok{\# Take the user bias values and the difference from the raw mean}
\CommentTok{\#user\_avg$user\_bias \textless{}{-} user\_avg$rating {-} global\_mean}

\CommentTok{\#user\_avg}

\CommentTok{\# Find the mean for all items in the training data}
\CommentTok{\#restaurant\_avg \textless{}{-} aggregate(rating \textasciitilde{} restaurant, data = train\_df, mean)}

\CommentTok{\# Take the item values and the difference frm the raw mean}
\CommentTok{\#restaurant\_avg$restaurant\_bias \textless{}{-} restaurant\_avg$rating {-} global\_mean}

\CommentTok{\#restaurant\_avg}
\end{Highlighting}
\end{Shaded}

\begin{Shaded}
\begin{Highlighting}[]
\CommentTok{\#Baseline Predictor}

\CommentTok{\# Merge regularized user and restaurant bias into training set}
\NormalTok{train\_df }\OtherTok{\textless{}{-}} \FunctionTok{merge}\NormalTok{(train\_df, user\_bias\_df[, }\FunctionTok{c}\NormalTok{(}\StringTok{"user"}\NormalTok{, }\StringTok{"user\_bias"}\NormalTok{)], }\AttributeTok{by =} \StringTok{"user"}\NormalTok{, }\AttributeTok{all.x =} \ConstantTok{TRUE}\NormalTok{)}
\NormalTok{train\_df }\OtherTok{\textless{}{-}} \FunctionTok{merge}\NormalTok{(train\_df, restaurant\_bias\_df[, }\FunctionTok{c}\NormalTok{(}\StringTok{"restaurant"}\NormalTok{, }\StringTok{"restaurant\_bias"}\NormalTok{)], }\AttributeTok{by =} \StringTok{"restaurant"}\NormalTok{, }\AttributeTok{all.x =} \ConstantTok{TRUE}\NormalTok{)}

\CommentTok{\# Predict using regularized baseline model}
\NormalTok{train\_df}\SpecialCharTok{$}\NormalTok{pred\_baseline }\OtherTok{\textless{}{-}}\NormalTok{ global\_mean }\SpecialCharTok{+}\NormalTok{ train\_df}\SpecialCharTok{$}\NormalTok{user\_bias }\SpecialCharTok{+}\NormalTok{ train\_df}\SpecialCharTok{$}\NormalTok{restaurant\_bias}

\CommentTok{\# Repeat for test set}
\NormalTok{test\_df }\OtherTok{\textless{}{-}} \FunctionTok{merge}\NormalTok{(test\_df, user\_bias\_df[, }\FunctionTok{c}\NormalTok{(}\StringTok{"user"}\NormalTok{, }\StringTok{"user\_bias"}\NormalTok{)], }\AttributeTok{by =} \StringTok{"user"}\NormalTok{, }\AttributeTok{all.x =} \ConstantTok{TRUE}\NormalTok{)}
\NormalTok{test\_df }\OtherTok{\textless{}{-}} \FunctionTok{merge}\NormalTok{(test\_df, restaurant\_bias\_df[, }\FunctionTok{c}\NormalTok{(}\StringTok{"restaurant"}\NormalTok{, }\StringTok{"restaurant\_bias"}\NormalTok{)], }\AttributeTok{by =} \StringTok{"restaurant"}\NormalTok{, }\AttributeTok{all.x =} \ConstantTok{TRUE}\NormalTok{)}

\NormalTok{test\_df}\SpecialCharTok{$}\NormalTok{pred\_baseline }\OtherTok{\textless{}{-}}\NormalTok{ global\_mean }\SpecialCharTok{+}\NormalTok{ test\_df}\SpecialCharTok{$}\NormalTok{user\_bias }\SpecialCharTok{+}\NormalTok{ test\_df}\SpecialCharTok{$}\NormalTok{restaurant\_bias}

\CommentTok{\# Merge user bias into the training data set by user}
\CommentTok{\#train\_df \textless{}{-} merge(train\_df, user\_avg[, c("user", "user\_bias")], by = "user", all.x = TRUE)}

\CommentTok{\#Merger item into the training data set by item}
\CommentTok{\#train\_df \textless{}{-} merge(train\_df, restaurant\_avg[, c("restaurant", "restaurant\_bias")], by = "restaurant", all.x = TRUE)}

\CommentTok{\#train\_df}

\CommentTok{\#Baseline Predictor for Training data avg + user bias + item bias}
\CommentTok{\#train\_df$pred\_baseline \textless{}{-} global\_mean + train\_df$user\_bias + train\_df$restaurant\_bias}

\CommentTok{\# Repeat same steps for testing dataset}
\CommentTok{\#test\_df \textless{}{-} merge(test\_df, user\_avg[, c("user", "user\_bias")], by = "user", all.x = TRUE)}
\CommentTok{\#test\_df \textless{}{-} merge(test\_df, restaurant\_avg[, c("restaurant", "restaurant\_bias")], by = "restaurant", all.x = TRUE)}

\CommentTok{\#test\_df$pred\_baseline \textless{}{-} global\_mean + test\_df$user\_bias + test\_df$restaurant\_bias}
\end{Highlighting}
\end{Shaded}

\begin{Shaded}
\begin{Highlighting}[]
\CommentTok{\# Baseline Predictor RMSE}

\CommentTok{\#remove the NA in the row}
\NormalTok{train\_filtered }\OtherTok{\textless{}{-}}\NormalTok{ train\_df[}\SpecialCharTok{!}\FunctionTok{is.na}\NormalTok{(train\_df}\SpecialCharTok{$}\NormalTok{rating), ]}
\NormalTok{test\_filtered }\OtherTok{\textless{}{-}}\NormalTok{ test\_df[}\SpecialCharTok{!}\FunctionTok{is.na}\NormalTok{(test\_df}\SpecialCharTok{$}\NormalTok{rating), ]}

\CommentTok{\# Using the RMSE function taking the difference between rating and baseline predictor for training and test set}
\NormalTok{rmse\_train\_base }\OtherTok{\textless{}{-}} \FunctionTok{rmse}\NormalTok{(train\_filtered}\SpecialCharTok{$}\NormalTok{rating, train\_filtered}\SpecialCharTok{$}\NormalTok{pred\_baseline)}
\NormalTok{rmse\_test\_base }\OtherTok{\textless{}{-}} \FunctionTok{rmse}\NormalTok{(test\_filtered}\SpecialCharTok{$}\NormalTok{rating, test\_filtered}\SpecialCharTok{$}\NormalTok{pred\_baseline)}


\NormalTok{rmse\_train\_base}
\end{Highlighting}
\end{Shaded}

\begin{verbatim}
## [1] 0.8429098
\end{verbatim}

\begin{Shaded}
\begin{Highlighting}[]
\NormalTok{rmse\_test\_base}
\end{Highlighting}
\end{Shaded}

\begin{verbatim}
## [1] 1.020685
\end{verbatim}

\begin{Shaded}
\begin{Highlighting}[]
\FunctionTok{cat}\NormalTok{(}\StringTok{"Raw Average Model:}\SpecialCharTok{\textbackslash{}n}\StringTok{"}\NormalTok{)}
\end{Highlighting}
\end{Shaded}

\begin{verbatim}
## Raw Average Model:
\end{verbatim}

\begin{Shaded}
\begin{Highlighting}[]
\FunctionTok{cat}\NormalTok{(}\StringTok{"  RMSE Train:"}\NormalTok{, }\FunctionTok{round}\NormalTok{(rmse\_train\_raw, }\DecValTok{3}\NormalTok{), }\StringTok{"}\SpecialCharTok{\textbackslash{}n}\StringTok{"}\NormalTok{)}
\end{Highlighting}
\end{Shaded}

\begin{verbatim}
##   RMSE Train: 1.134
\end{verbatim}

\begin{Shaded}
\begin{Highlighting}[]
\FunctionTok{cat}\NormalTok{(}\StringTok{"  RMSE Test: "}\NormalTok{, }\FunctionTok{round}\NormalTok{(rmse\_test\_raw, }\DecValTok{3}\NormalTok{), }\StringTok{"}\SpecialCharTok{\textbackslash{}n\textbackslash{}n}\StringTok{"}\NormalTok{)}
\end{Highlighting}
\end{Shaded}

\begin{verbatim}
##   RMSE Test:  1.183
\end{verbatim}

\begin{Shaded}
\begin{Highlighting}[]
\FunctionTok{cat}\NormalTok{(}\StringTok{"Baseline Predictor Model:}\SpecialCharTok{\textbackslash{}n}\StringTok{"}\NormalTok{)}
\end{Highlighting}
\end{Shaded}

\begin{verbatim}
## Baseline Predictor Model:
\end{verbatim}

\begin{Shaded}
\begin{Highlighting}[]
\FunctionTok{cat}\NormalTok{(}\StringTok{"  RMSE Train:"}\NormalTok{, }\FunctionTok{round}\NormalTok{(rmse\_train\_base, }\DecValTok{3}\NormalTok{), }\StringTok{"}\SpecialCharTok{\textbackslash{}n}\StringTok{"}\NormalTok{)}
\end{Highlighting}
\end{Shaded}

\begin{verbatim}
##   RMSE Train: 0.843
\end{verbatim}

\begin{Shaded}
\begin{Highlighting}[]
\FunctionTok{cat}\NormalTok{(}\StringTok{"  RMSE Test: "}\NormalTok{, }\FunctionTok{round}\NormalTok{(rmse\_test\_base, }\DecValTok{3}\NormalTok{), }\StringTok{"}\SpecialCharTok{\textbackslash{}n}\StringTok{"}\NormalTok{)}
\end{Highlighting}
\end{Shaded}

\begin{verbatim}
##   RMSE Test:  1.021
\end{verbatim}

\begin{Shaded}
\begin{Highlighting}[]
\NormalTok{result }\OtherTok{\textless{}{-}}\NormalTok{ (}\DecValTok{1}\SpecialCharTok{{-}}\NormalTok{ (rmse\_test\_base}\SpecialCharTok{/}\NormalTok{rmse\_test\_raw)) }\SpecialCharTok{*} \DecValTok{100}

\FunctionTok{cat}\NormalTok{(}\StringTok{"  RMSE Test Increased or Decreased: "}\NormalTok{, }\FunctionTok{round}\NormalTok{(result, }\DecValTok{3}\NormalTok{), }\StringTok{"}\SpecialCharTok{\textbackslash{}n}\StringTok{"}\NormalTok{)}
\end{Highlighting}
\end{Shaded}

\begin{verbatim}
##   RMSE Test Increased or Decreased:  13.702
\end{verbatim}

The Raw Average Model uses the overall average rating to predict user
ratings, resulting in:

\begin{verbatim}
Training RMSE: 1.134

Testing RMSE: 1.183
\end{verbatim}

The Baseline Predictor Model, which adjusts predictions by adding user
and restaurant biases to the global average, resulted in:

\begin{verbatim}
Training RMSE: 0.843 (lower than raw average)

Testing RMSE: 1.021 (lower than raw average)
\end{verbatim}

The Baseline Predictor Model improves test RMSE by about 13.7\% compared
to the Raw Average Model, indicating better generalization on unseen
data. We incorporated regularization into our model to reduce
over-fitting of the data and this process actually decreased our
baseline predictor model's test and training RMSE since we shrunk
extreme bias values toward the global average.

The lower RMSE of the baseline predictor confirms it outperforms the raw
average model, indicating meaningful improvement. Sushi Spot and Curry
Corner have negative item biases, suggesting they are consistently rated
below average by users. Burger Barn and Taco Town on the other hand
received higher average ratings from users which indicates customer
satisfaction.

The users Alice and Ella will provide higher ratings compared to others
on average which are positive user biases. The negative user biases
would be David and Cindy.

\end{document}
